\documentclass[twocolumn]{article}
\usepackage[utf8]{inputenc}
\usepackage{changepage}
\usepackage{graphicx}
\usepackage{amsmath}
\usepackage{amssymb}
\usepackage{amsthm}
\usepackage{mathtools}
\usepackage{float}
\usepackage[siunitx]{circuitikz}
\usepackage{tikz}
\usepackage{pgfplots}
\usepackage[colorlinks=true, linkcolor=black, citecolor=black, urlcolor=blue]{hyperref}
\usepackage{listings}
\pgfplotsset{width=10cm,compat=1.9}
\usetikzlibrary{positioning, arrows.meta}
\usepackage[a4paper, total={6in, 10in}]{geometry}
\bibliographystyle{ieeetr}

\title{\textbf{PROJECT VOICE - Respiration}}
\author{\textbf{Grace Winchell}}

\begin{document}

\maketitle

\section{Bioimpedance Sensors Overview}

Bioimpedance sensors for measuring breathing present a promising avenue for research and application, particularly in monitoring physiological parameters pertinent to individuals with autism who are non-verbal. These sensors can facilitate non-invasive health monitoring, providing critical data that can be utilized in machine learning models to analyze cognitive patterns and health conditions. Through the assessment of respiratory patterns, bioimpedance sensors can enrich the understanding of complex thoughts in individuals with autism and their responses to various stimuli.


\section{Importance of Breathing Monitoring}

Monitoring breathing patterns is crucial as it reveals essential information regarding a person’s physiological state and emotional responses. Breathing rate (BR) is one of the vital signs used in physiological monitoring. Non-invasive sensors enable the measurement of breathing without direct contact, making them suitable for diverse applications, including those involving individuals with autism.\cite{Ahmed_Scully_Vaughan_Wilson_Ozanyan_2014} 

\section{Functionality of Bioimpedance Sensors}

Bioimpedance sensors work by measuring the electrical impedance of biological tissues, allowing for the assessment of physiological changes, including respiration. They are particularly effective in capturing subtle changes in breathing dynamics influenced by emotional and cognitive states.\cite{Truong_Kim_2024} This can provide significant insights into the inner workings and thoughts of individuals with autism. 

\subsection{Functionality of Bioimpedance Sensors}

Bioimpedance sensors operate on the principle of measuring the electrical impedance presented by biological tissues when an alternating electrical current is applied. The tissues' impedance varies based on their composition, including fluid and cell type, which allows the sensor to detect changes in biological states.\cite{Janusz_Roudjane_Mantovani_Messaddeq_Gosselin_2022} This technology is particularly useful in medical diagnostics and physiological monitoring.

\subsection{Assessment of Physiological Changes}

The ability of bioimpedance sensors to assess physiological changes manifests in various applications, including monitoring respiratory rates. These sensors can detect fluctuations in impedance caused by the expansion and contraction of the lungs during breathing, thus providing real-time feedback on respiratory patterns.\cite{Janusz_Roudjane_Mantovani_Messaddeq_Gosselin_2022} The non-invasive nature of these sensors makes them ideal for continuous monitoring without causing discomfort to the user.

\subsection{Capturing Breathing Dynamics and Insights on Cognitive State}

Bioimpedance sensors are known for their sensitivity in capturing subtle changes in breathing dynamics. These dynamics can be influenced by various factors, including emotional and cognitive states. For instance, anxiety may lead to rapid, shallow breaths, whereas calmness might result in slower, deeper breathing. \cite{Janusz_Roudjane_Mantovani_Messaddeq_Gosselin_2022} By monitoring these changes, researchers can gain insights into the emotional and psychological responses of individuals.

The capability of bioimpedance sensors to track changes in breathing patterns allows for deeper understanding into the cognitive functions of individuals who are non-verbal or have communication difficulties. Changes in breathing can serve as a non-verbal cue regarding their inner thoughts or feelings, providing caregivers and researchers with invaluable data to interpret their states of mind. Individuals with autism may express their thoughts and emotions through physiological responses rather than verbal communication, and bioimpedance sensors can serve as a bridge for understanding these expressions.\cite{Janusz_Roudjane_Mantovani_Messaddeq_Gosselin_2022}

\subsection{Examples of Breathing and Emotion Correlation}
Waveforms of breathing patterns have been found to be correlated to feelings of anxiety, depression, anger, stress, and other positive and negative emotions. The opposite has also been found, breathing can also affect emotions.\cite{Jerath_Beveridge_2020} Relaxation and calmness are usually correlated to slower, deeper breathing.\cite{Nasseri_Lagman_Simon_Zhang_Mednick_2021} Joy and excitement has been shown to correspond to faster rhythmic breathing. Sadness and depression relate to slower, more labored breathing. \cite{Crockett_2014}

\section{Advantages of Non-Contact Measurement}

Non-contact measurement technologies, such as bioimpedance sensors, offer significant benefits for monitoring populations that may have difficulties with traditional methods. Unlike wired sensors, these devices do not require the subjects to wear equipment that could provoke discomfort or anxiety, thus facilitating a more natural assessment environment.\cite{Matar_Kaddoum_Carrier_Lina_2021}

\subsection{Benefits for Sensitive Populations}

These non-contact tools are particularly beneficial for populations such as individuals with autism, who may experience discomfort or anxiety in response to traditional monitoring methods. Many individuals may find wired sensors restrictive or uncomfortable, which can lead to increased stress or an unwillingness to cooperate during assessments. By minimizing physical interactions and discomfort, non-contact technologies help create a smoother and less stressful monitoring process. With non-contact measurements, these individuals can still engage in their regular activities in their own familiar surroundings. In these conditions, the subject will be more comfortable, yielding more accurate data about their physiological state.\cite{Boiko_Madrid_Seepold_2023}

\section{Application of Machine Learning with Bioimpedance Data}

The data collected from bioimpedance sensors can serve as input for machine learning algorithms. By analyzing respiratory patterns and their correlations with neurological responses, researchers can uncover nuanced insights into the cognitive and emotional states of non-verbal individuals with autism.\cite{Serrano-Finetti_Hornero_Casas_2022} This integration of data could lead to better tailored interventions, ultimately enhancing quality of life.

\section{Challenges and Future Directions}

While bioimpedance sensors present opportunities for growth in this field, several challenges must be addressed, including the need for accurate calibration and the development of robust algorithms to interpret complex data. Future research could focus on improving sensor designs, enhancing data accuracy, and exploring multi-modal approaches that combine bioimpedance data with other biometric signals for a comprehensive understanding of the cognitive processes in individuals with autism.

\section{Case Studies and Evidence}

Recent studies indicate that wearable sensors, including bioimpedance devices, have successfully monitored breathing patterns under various conditions.\cite{Yang_Huang_Weng_Liu_Chen_2024} They have proven effective for individuals undergoing physical activities, showcasing their potential for broader applications in health monitoring and cognitive assessment. 
By utilizing bioimpedance sensors for measuring breathing, researchers can gain deeper insights into the physiological and cognitive functions of non-verbal individuals with autism, promoting a better understanding of their complex thoughts and health needs.

\bibliography{citations}

\end{document}