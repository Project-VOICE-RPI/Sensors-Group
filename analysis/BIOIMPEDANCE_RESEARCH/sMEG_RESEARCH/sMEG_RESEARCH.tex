\documentclass[twocolumn]{article}
\usepackage[utf8]{inputenc}
\usepackage{changepage}
\usepackage{graphicx}
\usepackage{amsmath}
\usepackage{amssymb}
\usepackage{amsthm}
\usepackage{mathtools}
\usepackage{float}
\usepackage[siunitx]{circuitikz}
\usepackage{tikz}
\usepackage{pgfplots}
\usepackage[colorlinks=true, linkcolor=black, citecolor=black, urlcolor=blue]{hyperref}
\usepackage{listings}
\pgfplotsset{width=10cm,compat=1.9}
\usetikzlibrary{positioning, arrows.meta}
\usepackage[a4paper, total={6in, 10in}]{geometry}
\bibliographystyle{ieeetr}

\title{\textbf{PROJECT VOICE - sMEG}}
\author{\textbf{Grace Winchell}}

\begin{document} 

\maketitle

\section{sMEG Overview}
sMEG, surface electromyography, is a non-invasive technique utilized to assess and record electrical signals generated by skeletal muscles during contraction and relaxation. It involved putting electrodes over the target muscles, which detect the electrical potential as the muscle fibers are activated. 

\section{sMEG Mechanism}
Muscles contract when the motor neurons send electrical impulses to muscle fibers, causing them to depolarize and generate electrical activity. sMEG electrodes pick up these signals, which are then filtered and amplified for analysis.\cite{Manjarres-Triana_Acevedo-Serna_Ram’irez-Duque_Jim’enez_Pulido-Herrera_Mayor_2023}
\subsection{Electrodes}
Different electrodes can be used. Some electrodes, known as wet electrodes, use a conductive gel to improve skin contact. This improves the signal quality by reducing impedance. Dry electrodes may be more sensitive to skin impedance variations. 
\cite{Campanini_Merlo_Degola_Merletti_2007}
\subsection{Signal Analysis}
Signals collected are usually noisy. This can be due to other electrical equipment, other muscles based on where the electrodes are placed, and environmental factors. Different filters are used to remove this noise to enhance the signal. \cite{9749945}

\section{Muscle Response and Stress}
When individuals experience stress/anxiety, the body activates the "fight or flight" response. This includes increased muscle tension, which usually manifests in the jaw, shoulders, and neck. Muscles may involuntarily contract and remain that way after the stressors has passed. \cite{Blase_Vermetten_Lehrer_Gevirtz_2021}

\subsection{sMEG and Stress}
sMEG can be used to pick up on these muscle contractions and their intensity. By analyzing these signals, researchers and clinicians can determine the level of muscle tension at any given moment. Clinicians can compare pre- and post-intervention muscle activity to evaluate changes in stress response and muscle tension. \cite{Blase_Vermetten_Lehrer_Gevirtz_2021}
\section{Emotions and Facial Muscles}
Many feelings often trigger a involuntary muscle response. These are commonly seen in the face, neck, and shoulders. Different facial expressions come from different emotions, and these expressions use different facial muscles. The zygomaticus major muscle shows increased electrical activity when an idividual is smiling, meaning usually they are feeling happiness. The corrugator supercilii, which furrows the brow, may activate when an individual is angry. By placing electrodes over these muscles, emotions from facial muscle contractions can be read. 
\cite{Mamieva_Abdusalomov_Kutlimuratov_Muminov_Whangbo_2023}
\cite{Awana_Singh_Mishra_Bhutani_Kumar_Shrivastava_2023}
\subsection{Subtle Emotion Change Detection}
While the activation of certain muscles does correlate to different emotions. It takes looking at multiple muscles and their degrees of activation to read specific emotions. Research indicated that sEMG can categorize emotions beyond basic feelings, such as differentiating between happiness and subtle expressions of contentment or satisfaction. \cite{Sato_Murata_Uraoka_Shibata_Yoshikawa_2021} Another study also found that sMEG can discriminate between different positive emotions. 
\cite{Diederiks_2021}
\section{sMEG Comparison with Other Methods}
sMEG captures the electrical activity of facial muscles that are directly related to specific emotions, and has been shown to be accurate in showing small emotion variations
\subsection{Facial Expression Analysis}
Facial coding systems or computer vision-based analyses, also assess facial expressions but often rely on algorithms that can misinterpret subtle emotions due to variations in individual expressions or external factors, such as lighting conditions or camera angles. \cite{Künecke_Hildebrandt_Recio_Sommer_2014}
\subsection{Physiological Measures}
 Methods like heart rate variability (HRV), electrocardiography (ECG), and galvanic skin response (GSR) are also utilized for emotion detection, but can sometimes indicate more generalized states, such as arousal versus calmness. For example, while HRV and ECG can indicate stress levels, they may not differentiate between specific emotions like contentment and happiness as effectively as sEMG, which directly measures facial muscle contractions indicative of those feelings. 
 \cite{Mavratzakis_Herbert_Walla_2016}
 \cite{Pourmohammadi_Maleki_2020}
\section{Possible Limitations with sMEG}
The quality of the signal is very important to reading emotions, and this can be affected by many factors.
\subsection{Signal Reliability}
sEMG recordings can be contaminated by external noise and artifacts, particularly at low levels of muscle activity. These interferences can lead to inaccurate measurements, complicating the interpretation of results. \cite{Naik_Kumar_Arjunan_Palaniswami_Begg_2006} The skin also needs to be properly prepared for the electrodes, so without oils, hair, or dirt.\cite{Djuwari_Kumar_Arjunan_Naik_2008}
\subsection{Muscle Activation Detection}
Signals from adjacent muscles can cause ambiguity in muscle activation detection. 
The depth of muscles is also a problem, as most research has been done on surface muscles.\cite{Zheng_Wan_Xu_Wang_Qiao_2020}
\subsection{Subject Variation}
Differences in physical attributes, such as skin thickness, muscle mass, can cause changes in sMEG results. This makes it difficult to create a universal benchmark for muscle activation interpretation. \cite{Djuwari_Kumar_Arjunan_Naik_2008} Emotions are also personal, and so their impact on muscles can vary person to person.\cite{ceramic-on-metal_bearing_hip_replacement_of_Sciences_2023}
\bibliography{citations}
\end{document}