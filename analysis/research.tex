\documentclass{article}
\usepackage{graphicx} % Required for inserting images

\title{List of Existing Sensors}
\author{Tahira Tariq}
\date{September 2024}

\begin{document}

\maketitle

\section{Introduction}

\textbf{1. Eye Movement Sensors:}

\textbf{Electrooculography}

\textbf{(EOG)}: 

Uses electrodes placed around the eyes to capture voltage changes as the eye moves.

Mechanism:

Measures the corneo-retinal standing potential to detect eye movement.

DEF: 

Standing Potential: 

The difference of electrical potential of the anterior (front) and posterior (back) parts of the eyeball.

\textbf{Infrared Oculography}

\textbf{(IROG)}: 

Uses infrared light to track the position of the pupil and corneal reflection, 

providing detailed data on eye movement patterns.

\textbf{Video-Based Eye Trackers}: 

Uses cameras to capture eye movements by detecting the pupil and corneal reflections in real-time.

Mechanism:

Tracks the reflection of infrared light from the cornea and pupil to determine the point of gaze and eye movement.

\textbf{2. Heart Rate Sensors:}

\textbf{Electrocardiography }

\textbf{(ECG)}: 

Uses electrodes placed on the skin to detect the electrical activity of the heart. It is highly accurate but requires good skin contact and biocompatible materials to avoid skin irritation.

\textbf{Photoplethysmography }

\textbf{(PPG)}: 

Measures changes in blood volume using light absorption. It consists of a photodetector and light-emitting diodes (LEDs), which can be embedded in wearables like smartwatches for continuous monitoring [8].

\textbf{3. Body Heat and Skin Conductance Sensors:}

\begin{itemize}
    \item \textbf{Thermocouples and Thermistors}: These sensors measure temperature changes on the skin surface and can be used to detect body heat variations.
    \item \textbf{Electrodermal Activity (EDA) Sensors}: Measure skin conductance by detecting the electrical conductance changes in response to sweat gland activity. This method is commonly used for stress monitoring [6].
    \item \textbf{Flexible Wearable Sensors}: Incorporate materials like polypyrrole (PPy) and PEDOT

to detect various physiological signals including temperature and skin conductance [8].
\end{itemize}

\textbf{4. Sensor Mechanisms and Specifications:}

\begin{itemize}
    \item \textbf{Electrochemical Sensors}: Utilize changes in electrical properties like current, potential, and impedance to detect biochemical markers in sweat, saliva, or other body fluids. They are compact, cost-effective, and suitable for continuous monitoring [7].
    \item \textbf{Nanomaterials}: Carbon-based materials like graphene oxide and metallic nanoparticles enhance the sensitivity and electron transfer capabilities of sensors, making them more efficient for detecting subtle physiological changes [7].
    \item \textbf{Schottky Contact Sensors}: Convert mechanical stimuli directly into electrical outputs without the need for additional rectifiers, allowing for miniaturization and real-time body motion tracking [8].
\end{itemize}
 

\end{document}